\section{Gaussian Approximation of the Beam Waist}
Recall that previously, we used two diameters and the Gaussian power transmission through an aperture to find (in both cases) what our predicted beam width is $W_0=4.00mm$. Now, given an infinity-focused beam (i.e. Rayleigh length $>>1$) of a given width, we should be able to compute the new beam waist at the focal plane after it has passed through a lens using the following equation (Fundamentals of Photonics, 2nd ed., 3.2-13)
\begin{gather}
    W_0'= \frac{W_0}{\sqrt{1+(z_0/f)^2}}\\
    z' = \frac{f}{1+(f/z_0)^2}
\end{gather}
Note in our case, the Rayleigh length of the Gaussian laser beam $z_0>>1$, so we can instead use 3.2-17 to see
\begin{gather}
    2W_0' \approx \frac 4 \pi \lambda F_\#\\
    2W_0' \approx \frac 4 \pi \lambda \frac f D\\
    2W_0' \approx \frac 4 \pi \lambda \frac f {\diameter_{BFP}}
\end{gather}
I'm not certain what to implement for the excitation beam waist at the BFP. Our beam waist throughout the optical system is 
At the focal plane of the 125mm lenses, we have
\begin{gather}
    2W_0' \approx \frac 4 \pi \lambda \frac{125mm}{2(4mm)}\\
    2W_0' \approx 20 \lambda\\
    2W_0' = 22.3 \mu m
\end{gather}
for 561nm light. This would be easy to test via implementing a pinhole and checking to see if the power percentage transferred through the aperture is consistent with anticipated power transmission through a pinhole.\\
For for a 60X TIRF objective with a 200mm tube lens, we can do the Focal Length by doing $F_{TL}/M = 200mm/60 = 3.33mm$ so lets just go with 3.3mm. So we have something like this:
\begin{gather}
    2W_0' \approx \frac 4 \pi \lambda \frac{3.33mm}{2(4mm)}\\
    2W_0' \approx .297 \mu m
\end{gather}
Note that with a different formulation
\begin{gather}
    W_0' = \frac{f \lambda }{\pi W_0} 
    W_0' = \frac{3.33mm*561nm}{\pi 4mm}
    W_0' = 148.7 nm
\end{gather}
so we are being consistent at least. This seems unreasonably small, so that either our effective beam waist is far too large or something else odd is going on. \\

Note we have another possibility: using the Abbe diffraction limit as done above.
\begin{gather}
    \omega_0 \approx \frac{0.85 \lambda}{2 NA_{eff}}
    \omega_0 \approx 165nm
\end{gather}
This is a bit wider than we anticipated, but not as wide as we may have hoped. But here we have assumed that $NA_{eff} \approx NA_{theoretical}$ which may not be true. We need to find a more reasonable NA value. But how may we do this? I think one way is to shine a point in the BFP and see what comes out (angle wise). I think another is to assume that the LS is as wide as the effective NA allows it to be and see how that turns out. (Although this assumes a lot about the light sheet being effectively 1 point thick at the BFP.)
$NA = n\sin(\theta)$ so if we find the length of the LS we can figure out the effective NA. A quick gut check shows that we go between 10-105um for the LS height, so around 95 microns across. This gives an effective angle (arctan(a) = 47.5um/3.33mm)



Note, however, that this is the $W_0$ which is defined as the point where the intensity of the beam has fallen to $1/e^2$ of its maximum. Lets convert to $W_{FWHM}$ to make things simpler. We know that for Gaussian Beams, $W_0=0.843218*FWHM$, so $FWHM = .176\mu m$.
Its necessary to point out that Chad claims (and I'm sure he is correct) that \textbf{Gaussian Optics does not hold for objectives}. He does some rather complicated math to attain:
\begin{gather}
    FWHM_A=4.81 \mu m\\
    FWHM_L = 771nm
\end{gather}
for a Gaussian lightsheet in the axial and lateral directions respectively.
Chad sites the following in his computations (see p41 of his thesis).
    Wolf, E., Electromagnetic Diffraction in Optical Systems .1. An Integral Representation of the Image Field. Proceedings of the Royal Society of London Series a-Mathematical and Physical Sciences, 1959. 253(1274): p. 349-357.
203. Richards, B. and E. Wolf, Electromagnetic Diffraction in Optical Systems .2. Structure of the Image Field in an Aplanatic System. Proceedings of the Royal Society of London Series a-Mathematical and Physical Sciences, 1959. 253(1274): p. 358-379.
204. Chon, H.S., et al., Dependence of transverse and longitudinal resolutions on incident Gaussian beam widths in the illumination part of optical scanning microscopy. J Opt Soc Am A Opt Image Sci Vis, 2007. 24(1): p. 60-7.

