\maketitle
\section{Principles of Remote Focusing}
\subsection{Introduction}

\subsection{Remote Focusing and Transport-of-Intensity Fourier Optics}
Note this is document is based largely on \textit{High-speed transport-of-intensity phase
microscopy with an electrically tunable lens} by Zuo et al.\\
We assume our system is telecentric, which guarantees that refocusing avoids the introduction of additional phase-curvature across the field of view. This would result in the addition of variable magnification over the field of view and z-stack. This is crucial, as otherwise data processing would be unduly arduous and Under this assumption, we can model the propogation of the wavefield $u_0$ modified by the free speace transfer function $H$ as follows. 
\begin{gather}
	u_{\Delta z}(x,y) = \mathcal{F}^{-1}\left [ \mathcal{F}(u_0(x,y))H_{\Delta z}(u,v)\right]
\end{gather}
In brackets, we see that we have taken the Fourier transform of the wave to find its corresponding frequency-space function, and then modify with the transfer function. In the paraxial approximation, the free space transfer function is known to be
\begin{gather}
	H_{\Delta z}(u,\nu) = \exp\left[-i \pi \lambda \Delta z (u^2 + \nu^2)\right].
\end{gather}
We have ignored the phase factor as it is invariate for our purposes. Replicating the phase modificatin of free space via a spatial light modulator (SLM) represents the angular spectrum method and is widely used in holographic reconstruction wdue to its ability to produce an image field which does not vary in size along the optical axis. However, there are many practical difficulties of working with an SLM, we instead utilize a electronically tunable lens (ETL). In Fourier optics, in the paraxial approximation a lens produces a transfer function of the form (CITE Goodman eq. 5-10)
\begin{gather}
	t_l(x,y) =\exp \left[ - i \frac{k}{2f}\left(x^2 + y^2\right)\right] .
\end{gather}
Since our ETL is in the spatial coordinates of the Fourier plane of the 4f system, it has a transfer function of
\begin{gather}
	t_l(\xi, \eta) = \exp\left[ \frac{-i \pi}{\lambda f_c}\left(\xi^2 + \eta^2\right)\right].
\end{gather}
Notice that equation 4 is similar in form to the free space transfer function given in equation (2); by selecting a specific combined focal length $f_c$, we can do the equivalent of propogating the wave through free space some distance $\Delta z$. The combined focal length may be manipulated by changing the focal length of the ETL, which may be done by passing a current through the liquid crystal. We can therefore select the effective distance our wave propogates by changing the focal lenght of the ETL, allowing us to electroncially increase or decrese the effective optical length of our path. The combined focal length of the ETL and the normal lens may be expressed as a function of the ETL's focal length, the optical lens' focal length, and the distance between them d. We may now express this in terms of $(u, \nu)$ and then compare it to our known free space transfer function $H_{\Delta z}$ above, then solve for $\Delta z$. Note that in this case we convert from Foruier spatial coordinates to frequency coordinates with {CITE GOODMAN} $ (\xi, \eta) \to (\lambda f u, \lambda f \nu)$ to see 
\begin{gather} 
		H_{\Delta z}(u,\nu)=t_l(\xi, \eta) \\
		 \exp\left[-i \pi \lambda \Delta z (u^2 + \nu^2)\right] = \exp\left[ \frac{-i \pi}{\lambda f_c}\left((\lambda f u)^2 + (\lambda f \nu)^2\right)\right]\\
		\lambda \Delta z (u^2 + \nu^2) = \frac{\lambda f^2}{f_c}(u^2+\nu^2)\\
		\Delta z = \frac{f^2}{f_c}
\end{gather}
Note that the authors of {HIGH SPEED TRANSPORT CITE} give
\begin{gather}
	f_c = \frac{f_{ETL}f_{OL}}{f_{ETL}+f_{OL}-d}
\end{gather}
In our case, however, we use only a stand-alone ETL rather than a combined system. Specifically, the light passes through a flat pane of glass before hitting the ETL, so the focal length of our optical lens goes to infinity. Since the optical lens focal length blows up to infinity, it dominates the denominator and cancels with itself in the numerator so that
\begin{gather}
	 \lim_{f_{OL} \to \infty} f_c = f_{ETL}
\end{gather}
which simplifies our calculations significantly. We must also account for the magnification of our system, which reduces the axial range of our remote focusing. To understand why, we turn to an intuitive ray optics approach.

\subsection{Using Ray Optics to find Axial Scanning Range}
\textbf{Cite Qu, Hu!} When centered at the focal plane of two relay lenses rather than at the back focal plane, ETL3 also causes minor demagnification at high dioptric power. To see why this is, we turn to another paper which analyzes the impact of a remote focusing objective in the detection pathway in more detail. We begin with the thin lens equation
\begin{gather}
	\frac{1}{I} + \frac{1}{I'} = \frac 1 f
\end{gather}
for $f_O$ is the focal length of the objective, $I$ is the distance between the object and the objective and $I'$ is the distance between the objective and the image. Now let us assume that we have an ETL at the back focal plane of our objective. This ETL produces an axial translation of the focal plane $\Delta z$. This means we may now write a system of two thin lens equations to model the light going from the object plane, through the objective, to the BFP where the ETL is, and then to the imaging plane.
\begin{gather}
	\frac{1}{I-\Delta z} + \frac{1}{I_1'}  = \frac 1{f_O'}\\
	\frac{1}{f_O' -I_1'}+\frac{1}{I_2'} = \frac{1}{f_e'}
\end{gather}
The first thin lens equation models the system from the new object plane, which has been shifted by $\Delta z$ from the old object plane via remote focusing by the ETL, through the objective, and to the image after the objective at a distance of $I_1'$. The second thin lens equation models the system from the image after the objective through the ETL ofvariable focal length $f_e$ and to the second image after the ETL at a distance of $I_2'$. Due to the configuration of our system, the distance between the image after the objective and the objective is equal to the distance from the objective to the ETL at the back focal plane, which is the focal length, and the distance from ETL2 to the image after ETL2 where the camera is placed. Therefore,
\begin{gather}
	I' = I_2' + f_0'.
\end{gather}
Now, we can solve for this system of equations. It is tedious, so I include a MATLAB file to compute the solution rather than reproducing it here. We can solve for $\Delta z$ to give
\begin{gather}
	\Delta z = \frac{f_0'^2}{f_e'}
\end{gather}
Such an ETL does not alter the mangification of the system. However, in our system, ETL2 and ETL3 are at the focal plane of two relay lenses conjugate to the back focal plane rather than in the back focal plane itself. In this case, we get a thin lens eqaution describing the transition from object through the objective to the image and then from the intermediate image through the ETL to the detector. We see
\begin{gather}
	\frac{1}{I-\Delta z}+\frac 1{I_1'} = \frac 1{f_0'}\\
	\frac{1}{I' + f_r' - I_1'}+\frac1{I_2'} = \frac{1}{f_r'}
\end{gather}
 where variables are unchanged from before except $I_2'$ is the distance between the image afer the first relay lens and the relay lens, and $f_r'$ is the focal length of the relay lenses. Also note that due to our setup, we know that there is an image in the front focal plane of the ETL after the relay lens, so if the ETL is a distance $d_1$ from the relay lens, we get
\begin{gather}
	d_1 = I_2' + f_e'.\\
\end{gather}
The distance from the relay lens to the ETL is given by 
\begin{gather}
	d_1 = f_r' + \frac{f_2'^2}{M_0f_0'}
\end{gather}
To determine the relationship between these, we again solve using MATLAB, finding
\begin{gather}
	d_1
\end{gather}
\newpage

\section{Callibration Procedure}


\subsection{Introduction to our Optical System}

Our optical system is designed for light sheet fluorescence microscopy, and has two paths which diverge at the dichroic mirror of a filter cube just after a shared objective. 
In our system, one remote focusing lens, ETL 2, is responsible for remote focusing in the illumination light path, while the other remote focusing lens, ETL3, is responsible for focusing in the detection light path. In our field of research, it is important to be able to translate the focal plane of the light sheet by a set amount using ETL2, while ETL3 must be able to move the imaging plane to the same location.  In order to use these devices as designed, we must therefore create a look up-table for each ETL, tying applied current to optical axis translation$\Delta z$. \\
% NOT NEEDED: In the case of the illumination path, we will measure the beam waist of a light sheet in the focal plane, then define the Rayleigh length of the light sheet.
Going forward, it will be important to differentiate between the illumination focal plane, the sample plane, and the detection focal plane. The \textit{illumination} focal plane is the plane in which the beam waist lies, and is translated via altering the dioptric power of ETL2 or physically moving the objective. The sample plane is the physical location of the sample's surface, and is static for any given experiement. The \textit{detection} focal plane is the plane which is imaged on the CCD, and is translated via altering the dioptric power of ETL3 or physically moving the objective.\\

% OUT OF PLACE:
% Approximating the lightsheet as Gausssian in its thin direction, we will then use this information to translate the sample stage upwards in known increments, using the illumination ETL to keep it focused on the sample slide. I also wish to quantify the minimum beam waist of the light sheet at various optical displacements and compare it to the Abbe Diffraction limit and the Gaussian diffreaction limits, as this is of interest to our experimental design for this system.\\
 In theory, both the remote focusing ETLs should have a relationship which ideally will be expressed by equation 8, which predicts  that remote focusing will cause a smooth, linear translation in the optical power as we change the dioptric power, which ideally is itself linearly linked to applied current. Since current is the only thing we actually control, the non-ideal behavior of the ETL compounds with the non-ideal behavior of the optics so that in practice behavior may differ significantly from the ideal, and computations end up being more of a rule-of-thumb than an absolute, all-in-one solution.  In practice, however, I predict that the ETLs will have slightly different properties, since ETL2 occurs \textit{before} the objective in the illumination light path, while ETL3 occurs \textit{after} the objective in the detection light path. Theoretically, this shouldn't matter much, but with imperfect alignment and non-ideal optics, I predict there will be small differences in thier behavior. Furthermore, based on the literature, I predict that the lenses will behave worse at high dioptric power, and that high ETL3 dioptric power will cause minor image demagnification.

\subsection{ETL3}
Insert!
\subsection{ETL2}

ETL2 is in the detection pathway of Hestia. As a result, although we are only interested in calibrating the illumination pathway, we must be able to see what we are doing, so we must use both the illumination and detection pathways. Using either a spot or light sheet, we will first focus the illumination beam into the sample plane via objective height adjustment and ETL2.

It is important that we select an objective which has a large enough working distance and thus a sufficiently versatile z-axial domain to enable focused imaging along the full dioptric range of ETL2 without, for instance, bumping into our slide.

 We will use microManager to move the microscope objective upwards in the Z direction in known discrete steps using the "Manual Focus Position" setting, which moves the objective to a given z-coordinate.  We will then bring this image into focus in the camera via ETL3 to adjust for the axial displacement and keep the image focused. We may then minimize the width of the light-sheet or beam by translating the location its focal plane in the Z-axis to our current plane via ETL2. Because we have dictated the $\Delta z$ step the objective has taken, we know how much ETL2 has translated the focal plane, enabling us to link ETL2 current to focal-plane translation.

In this way, we can map ETL2 current onto axial displacements. Unfortunately, due to the lack of an "autofocus" feature in the ETL, and no way for the camera and ETL actions to synchronize effectively in our current software configuration, this must be done manually.
It is very important that we ensure that the change tilt angle $\Delta \theta$ of our light-sheet is as small as possible during this calibration. This is because significant tilt angles trigonometrically increase the effective path the light-sheet must travel after exiting the objective. Variations in the tilt angle result in an effective change in the focal plane of the illumination light $\Delta z$ independent of the change in ETL dioptric power, introducing a confounding variable and messing up our LUT.

% Below, paste the light sheet computations! But first, update them for the correct effective NA.

CITE Grewe, B.F., et al., Fast two-layer two-photon imaging of neuronal cell populations using an electrically tunable lens. Biomed Opt Express, 2011. 2(7): p. 2035-46.