\section[]{Fiolka Light Sheet Beam Waist Method}
subsection{FIOLKA Lightsheet Computations}\\
Note the primary source for this work is Fiolka's Lightsheet literature review. The beam waist in a LSFM system is given by 
\begin{gather}
    \omega_0 \approx \frac{0.85 \lambda}{2 NA}
\end{gather}
    Where NA is the NA of the illumination objective and $\lambda$ is the excitation wavelength. The lateral beam radius is given by 
    \begin{gather}
        \omega(x) = \omega_0 \sqrt{1 + \left(\frac{x}{x_R}\right)^2}\\
        x_R = \frac{n \pi \omega_0^2}{\lambda}
    \end{gather}
    where $x_R$ is the Rayleigh length in the lateral direction.
    In the y-direction, the beam is essentially of infinite width and thus limitless depth of focus, as this is the broad side of the sheet. The only limitation here is the optical train componenets.\\
    The lateral resolution of LSFM (i.e. in the x-y plane) is defined by 
    \begin{gather}
        \Delta r = \frac{\lambda_{em}}{2 NA_{det}}
    \end{gather}
    for the emission and detection wavelengths. This is because this is the FWHM radius of the beam, which is the point at which two airy disks would reach their first trough and be distinguishable.\\
    The axial resolution is given by the product of the illumination and detection point spread function, which in turn depend on the excitation and emission wavelengths adn the NA of the objective (which in our case is combined). In general, thinner light-sheets are generated by higher NA objectives with higher axial resolution but also have narrower Rayleigh lengths and thus worse field of view. In other words, it is messy, but can be approximated (assuming $NA_{exc}=NA_{ill}$) by
    \begin{gather}
        \Delta z \approx \left( \frac{2 NA_{ill}}{\lambda_exc} + \frac{n (1-\cos \theta)}{\lambda_em}\right)^{-1}
    \end{gather}
    where $\theta$ is the half-angle of the detection objective
    \begin{gather}
        \theta = \arcsin\left(\frac{NA}{n}\right)
    \end{gather}
    Now, we must compute the ideal beam width and compare this with our own findings. For our TIRF objective, which has an NA of and 561nm laser light,
    \begin{gather}
        \omega_0 \approx \frac{0.85\cdot561nm}{2*(1.5)}\\
        \omega_0 \approx 158.95nm
    \end{gather}
We can also use a beam simulator to verify our result--see https://github.com/remachae/beamsimulator. We end up with a main lobe width of .32$\mu m$, which is equivalent to $\omega_0=160nm$; our answer checks out.\par
Notice we also have a lateral \cite{Ernst} resolution limit of 
\begin{gather}
    \Delta r = \frac{\lambda_{em}}{2 NA_{det}}
    \Delta r = .187 \mu m
\end{gather}
\subsection{Considering a Tilted Light-Sheet}
We have to keep in mind that our light sheet is actually at some non-trivial angle. Considering the beam at an angle $\theta$, we consider the measured beam waist $W_0'$ to be the hypotenuse, the actual beam waist $W_0$ (i.e. what we would see if the beam were aligned vertically) is offset by some angle $\theta$, and the transverse distance traversed over the beam's angled path is orthogonal to the actual beam waist. We then have 
\begin{align}
    \cos \theta = \frac{W_{0}}{W_0'}\\
    W_0 = W_0' \cos \theta 
\end{align}
Note that the entire beam is being shifted upwards by translating the objective, so that as long as the angle of the beam approaching the objective is very small (and it certainly ought to be) there is not any sort of trigonometric consideration regarding the pre-objective beam angle. Now, we can correct for the beam waist.
\subsection{Examining a tilted light-sheet in the back focal plane}
Consider a beam entering the back focal plane. In order to make a light sheet, the beam must form a band following a narrow linear path across the BFP of the objective. In order to avoid lateral tilt, it must be centered in its narrow axis. If such a beam fully fills the back focal plane of the objective, it will create what is essentially an X, with all of the rays converging to the focal plane. Such a light-sheet cannot, however have any angle orthogonal to its length, because such a tilt is caused by being offset in one direction or another. Instead, it merely fills the entire possible azimuthal angle domain $\theta \in [-\theta_{max},\theta_{max}]$, where $\theta_{max}$ is the maximum possible angle allowed by $NA=\eta \sin \theta$. In other words, our light sheet is maximally angled, but is in the form of an upright hourglass rather than a tapering ramp. This is useless for reducing background, as we always will have a lot of background below the beam waist-focal plane intersection. In order to have a shape of the form $/|$ or something similar, the beam must be trimmed in the back focal plane so that we can angle both the bottom line and the top line, such that 

$\theta \in [\theta_1, \theta_2]$ for $0<\theta_1<\theta_2$ to form a tapering inclined ramp style light-sheet. In this case, we have an effective NA given by
\begin{equation}
    NA_{eff}=1.51 \sin (\theta_1-\theta_2).
\end{equation}
so that 
\begin{equation}
    w_0 = \frac{.85 \lambda}{2NA_{eff}}.
\end{equation}
From this, we can compute a new Rayleigh length
\begin{gather}
    Z_R = \frac{\pi w_0^2}{\lambda}\\
    Z_R = \frac{\pi (0.85)^2 \lambda}{4 (NA_{eff})^2}\\
    Z_R = \frac{\pi (0.85)^2 \lambda}{4 \left(1.51 \sin (\theta_1-\theta_2)\right)^2}\\
\end{gather}
So if we need the light sheet to be about $10 \mu m$, we can findings
\begin{gather}
    10\mu m = \frac{\pi (0.85)^2 \lambda}{4 (1.51 \sin (\theta_1-\theta_2))^2}
\end{gather}
\subsection[]{Tokunaga}
From Tokunaga, we see that
\begin{gather}
    R = dz \tan \theta
\end{gather}
for $R$ is the thickness of the light sheet's cross section with the coverslip.
To acquire a beam waist of the desired size at $\theta \approx 70^\circ$, which gives us $dz \approx 3.64 \mu m$.  Tokunaga gives 
\begin{gather}
    x=n  \times f_{obj} \times \sin \theta
\end{gather}
as his way of obtaining the angle $\theta$.
When $R / \tan \theta $ is small, Tokunaga notes $dz$ inflates more than predicted due to divergence of the illumination at the specimen, so to reduce this Tokunaga implements a field stop conjugate to the specimen plane. In order to minimize divergence at the edges of the field stop, it is conjugate to the specimen plane instead.
We can use simple gaussian beam propogations to determine the beam waist at a given location in the optical system. See my excel sheet for more information.