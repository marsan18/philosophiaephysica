\section{Fluorescence Math}
Let $I(\lambda)$ be equivalent to the cuumulative intensity of the image at the CCD while $Q(\lambda)$ is the quantum efficiency of the CCD. Then by integrating the efficiency times the intensity of the image over some timescale and the spectrum detectable by the sensor, we get:
\begin{gather}
    I(\lambda) = \Delta T [P(\lambda) Ex(\lambda) A D Fl(\lambda) Em(\lambda) I(\lambda) + N_{Optical}(\lambda)]\\
    C = \int_0^t \int_{\lambda} Q(\lambda) I(\lambda) d\lambda dt + N_{Readout}
\end{gather}
We now must define some sort of reasonable estimate for $I(\lambda)$. We start out with the power emission spectra of the fluorescence LED as a function of wavelength, $P(\lambda)$. We then multiply this by the fraction of light transmitted by the excitation optical train, including all mirrors, filtersets, and objectives, $T_{ex}(\lambda)$. The product of $F_{ex}(\lambda)$ and $P(\lambda)$ gives the true excitation spectrum $X(\lambda)$ at the fluorophore. We then muliply this by single fluorophore cross sectional factor $A$ and concentration factor $D$, finding the excitation's effective effect modifier on the specimen. In other words, the excitation power is given by 
\begin{equation}
	X(\lambda) = P(\lambda) T_{ex}(\lambda)
\end{equation}
Now, we know that the emission of the fluorophore $M$ is a function of both wavelength and the excitation spectrum, which in turn is a function of wavelength. Let $M_{eff}(\lambda)$ be the fluorescence efficiency factor as a function of excitation wavelength. So we have 
\begin{equation}
	M(\lambda, X(\lambda))  = X(\lambda) M_{eff}(\lambda) 
\end{equation}
Technically speaking this is all a function of wavelength; every function so far is a simple pairing of power (or factor giving the percentage of power transmitted) to wavelength. However, it is confusing to incorporate all of the excitation function into the emission function implicitly. It is better represented by examining the efficiency of the fluorophore and the excitation spectrum separately. Basically, all of this is just to say that $X(\lambda)$ determines the excitation at the specimen and $M_{eff}(\lambda)$ represents the fluorophore. We therefore have that the light emitted is
\begin{equation}
	M(\lambda)  =P(\lambda) T_{ex}(\lambda) M_{eff}(\lambda).
\end{equation}
Now, for the emitted light to get to the camera detector, we need to pass through the detection path. Let's represent the transmission of the detection pathway as $T_{em}(\lambda)$. Then the amount of emission power reaching the camera $I$ is given by the product of the emission transmission and the light emitted.
\begin{gather}
	I(\lambda) = T_{ex}(\lambda) M((\lambda)\\
	I(\lambda)= P(\lambda) T_{ex}(\lambda) M_{eff}(\lambda) T_{em}(\lambda)+N_{Optical}
\end{gather}
Where $N_{Optical}$ represents the optical noise present in experimental setups. \\

We now have to convert from the image at the CCD to the digital signal we actually capture for data analysis. Firstly, the camera has several effects. A monocolor sCMOS effectively integrates the signal over the detectable wavelength spectrum and has a quantum efficiency, $Q(\lambda)$, which varies across it's operating range. sCMOS cameras also have some amount of eletronic background noise (also called readout noise) $N_{Readout}$. To convert to a digital signal $C$, we integrate over the emission light times the camera's efficiency with respect to wavelength and time to get an expression for our digital signal as
\begin{gather}
	C = \int Q(\lambda)I(\lambda) d\lambda dt + N_{Readout}.
\end{gather}
So our entire expression may be written as 
\begin{gather}
   C = \iint Q(\lambda) \left(P(\lambda) T_{ex}(\lambda) M_{eff}(\lambda) T_{em}(\lambda)+N_{Optical}\right) d\lambda dt + N_{Readout}
\end{gather}
This is a bit nasty. However, if we use a laser to excite the fluorophores, things simiplify greatly. Assuming a known amount of laser power at $\lambda = \Lambda$, we see
\begin{gather}
            C=  \iint Q(\lambda) \left( T_{ex}(\Lambda) M_{eff}(\Lambda) T_{em}(\lambda)+N_{Optical}\right) d\lambda dt + N_{Readout}.
\end{gather}
This saves us from having to figure out the spectrum of excitation light at the specimen. In theory, this would be a simple measure of plugging a known excitation source into FPbase with the excitation filters and seeing the output, but in practice this could be quite hard since I can't find the Prismatix data. It also heads off any wavelength-dependent stokes shift phenomena and reduces our excitation mess to a simple measurement of the laser in the BFP of the objective followed by multiplying by an efficiency factor for the emission efficiency rather than integrating over some sort of list of values. So effecitvely,we can boil all of those excitation terms into some constant $K$.
\begin{gather}
            C=  \iint Q(\lambda) \left(K * T_{em}(\lambda)+N_{Optical}\right) d\lambda dt + N_{Readout}.
\end{gather}

